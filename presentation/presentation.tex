\documentclass{beamer}

\usepackage{hyperref}
\usepackage{listings}
\usepackage{color}

\usepackage[backend=bibtex]{biblatex}
\addbibresource{citations.bib}

% http://tex.stackexchange.com/questions/68080/beamer-bibliography-icon
\setbeamertemplate{bibliography item}{%
  \ifboolexpr{ test {\ifentrytype{book}} or test {\ifentrytype{mvbook}}
    or test {\ifentrytype{collection}} or test {\ifentrytype{mvcollection}}
    or test {\ifentrytype{reference}} or test {\ifentrytype{mvreference}} }
    {\setbeamertemplate{bibliography item}[book]}
    {\ifentrytype{online}
       {\setbeamertemplate{bibliography item}[online]}
       {\setbeamertemplate{bibliography item}[article]}}%
  \usebeamertemplate{bibliography item}}

\defbibenvironment{bibliography}
  {\list{}
     {\settowidth{\labelwidth}{\usebeamertemplate{bibliography item}}%
      \setlength{\leftmargin}{\labelwidth}%
      \setlength{\labelsep}{\biblabelsep}%
      \addtolength{\leftmargin}{\labelsep}%
      \setlength{\itemsep}{\bibitemsep}%
      \setlength{\parsep}{\bibparsep}}}
  {\endlist}
  {\item}

% Colours for beamer.
\setbeamercolor{frametitle}{fg=orange}
\setbeamertemplate{itemize item}{\color{orange}$\blacksquare$}
\setbeamertemplate{itemize subitem}{\color{orange}$\blacktriangleright$}

% Colours for syntax highlighting
\definecolor{syntax_red}{rgb}{0.7, 0.0, 0.0} % For strings
\definecolor{syntax_green}{rgb}{0.15, 0.5, 0.25} % For comments
\definecolor{syntax_orange}{rgb}{0.7, 0.4, 0.2} % For keywords


% Haskell settings for lstlisting
\lstset{language=Haskell,
basicstyle=\ttfamily\tiny,
keywordstyle=\color{syntax_orange}\bfseries,
stringstyle=\color{syntax_red},
commentstyle=\color{syntax_green},
numbers=none,
numberstyle=\color{black},
stepnumber=1,
frame=single,
breaklines=true,
numbersep=10pt,
tabsize=4,
showspaces=false,
showstringspaces=false}

\author{
  Beck, Calvin \and Huang, Jiani \and Li, Yishuai
}

\begin{document}

\begin{frame}
  \frametitle{FuzzChick}
  \maketitle
\end{frame}

\section{Introduction}

\begin{frame}
  \frametitle{What is this talk about?}

  \pause

  {\huge FuzzChick!}\\~\\

  \pause

  What is FuzzChick...? \\~\\

  \pause

  FuzzChick is an experiment to improve QuickChick using ideas from
  fuzzing, such as AFL.\\~\\~\\

  \pause

  {\huge We'll come back to this...}\\

\end{frame}

\begin{frame}
  \frametitle{QuickChick: A Brief Review}

  QuickChick is a properties based testing framework for Coq. \\

  \begin{itemize}
  \item You build (or derive) generators for data types.
  \item Using those generators you can feed data into test cases.
  \item These test cases can be any arbitrary predicate.
  \end{itemize}
\end{frame}

\begin{frame}
  \frametitle{QuickChick: Pros and Cons}

  So what's great about QuickChick?\\

  \begin{itemize}
  \item Relatively easy to build / derive generators.
  \item Can generate lots of tests for specific properties
    automatically.
  \end{itemize}

  \pause

  What's not so great about QuickChick?\\

  \pause

  \begin{itemize}
  \item Getting \emph{good} generators can be hard!
  \end{itemize}
\end{frame}

\section{Good Generators}

\begin{frame}
  \frametitle{What makes a good generator?}

  The basic idea of what makes a generator ``good'' can vary somewhat
  based on the context. \\

  \pause

  In general you want good coverage. How can you achieve that with
  minimal work?
\end{frame}

\section{FuzzChick}

\begin{frame}
  \frametitle{Finally, FuzzChick!}

  FuzzChick uses AFL to make the choices between constructors for
  building data types for tests.\\

  \pause

  Why is this good?

\end{frame}

\begin{frame}
  \frametitle{FuzzChick Intuition}

  AFL uses DSE to attempt to get good coverage while fuzzing...\\

  \pause

  Maybe we can utilize AFL's smarts to achieve better test coverage.
\end{frame}

\section{Evaluating FuzzChick}

\subsection{FuzzChick and Apache}

\subsection{FuzzChick Coverage}

\begin{frame}
  \frametitle{QuickChick: Now With Coverage!}

  We instrumented QuickChick using bisect\_ppx to get coverage
  estimates!\\~\\

  \pause

  This \emph{mostly} went smoothly...\\~\\

  \includegraphics[width=\textwidth]{bisectbug.png} \\

  \pause

  Maintainer fixed this issue promptly, which was \emph{awesome}!
  
\end{frame}

\begin{frame}
  \frametitle{QuickChick with coverage... Still needs some polish}

  QuickChick with coverage is cool, but it still does need some
  polish.\\

  \begin{itemize}
    \pause
  \item Includes a lot of excessive extracted Coq code.
    \pause
  \item QuickChick generates ``random'' files for each test, and the
    names aren't all that useful
  \end{itemize}

  \pause

  But...\\~\\

  \pause

  {\huge It works! We can measure stuff!}
\end{frame}

\section{End}

\begin{frame}
  \frametitle{Conclusion! Questions?}

  \huge{Whew! Questions?}
\end{frame}

\begin{frame}
  \frametitle{References}

  \nocite{*}
  \printbibliography

  These are all good resources! You should look at them!
\end{frame}
\end{document}
